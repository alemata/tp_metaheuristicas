\section{Metaheuristica Simple}
\label{sec:intro}

En nuestro trabajo implementamos una metaheurística de poblacion (Colonia de Hormigas) 
por lo tanto describiremos brevemente como podriamos implementar una técnica 
simple. 

En este caso elejimos la metaheristica $GRASP$ (Greedy Randomized Adaptive Search 
Procedure). Consiste en ir generando soluciones iniciales con un método voraz 
randomizado, tratar de mejorar las mismas haciendo una busqueda local e ir 
actualizando la mejor solucion. El pseudo-código general seria el siguiente:

\begin{Verbatim}[samepage=true]
main {
  Mientras no se verifica criterio de parada {
      solución = ConstruirSoluciónVoráz()
      mejorLocal = BusquedaLoca(solución)
      mejorSolución = CompararSoluciones(mejorLocal, mejorSolución)
  }
  devolver mejorSolución
}
\end{Verbatim}

Por lo tanto bastaria con definir como generamos las soluciones iniciales, y 
cual es la vecindad de una solución para realizar la busqueda local. 

Para generar las soluciones iniciales la idea es ir agregando digitos al azar, 
en los casilleros en los cuales haya menos posibiliades (posibles digitos a agregar dadas las restricciones del sudoku). 
Por ejemplo si en un casillero solo tenemos 2 digitos, elegimos alguno de ellos al azar. Asi vamos completando
hasta que no podamos agregar mas digitos. De esta manera, estamos generando 
soluciones random.


Una vez que obtenemos una solución debemos definir cual es su vecindad para poder ejecutar la busqueda local
sobre la misma. Una posible vecindad seria eliminar los digitos de las filas y columnas que no hayan sido totalmente
llenadas por el algoritmo (menos los digitos iniciales), y volver a llenarlos 
repitiendo los pasos del algoritmo inicial. 
Podemos estar seguros de que al menos una fila y/o columna no 
estará llena, debido a que si eso ocurre, es porque ya encontrarmos la mejor solucion posible.

Con esto tendriamos los pasos para implementar una heristica de GRASP. 

\newpage
