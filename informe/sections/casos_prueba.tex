\section{Casos de prueba}
\label{sec:intro}

Para evaluar el comportamiento y fijar los parametros del algoritmo utilizamos los casos de prueba obtenidos de
\url{http://lipas.uwasa.fi/~timan/sudoku/}, los cuales estan ordenados en cuantro categorías: \textit{fácil, medio, difícil y ultra difícil}.

Nuestra heuristica logro encontrar el resultado en todos los casos de prueba en 
un tiempo inferior a los 5 minutos. En las categorias  \textit{fácil y medio} no 
se llegaba a la primer iteracion del mismo.

Esto significaba que el algoritmo estaba encontrando la solucion sin compartir 
la informacion entra las hormigas. 

Por lo tanto para fijar los parametros decidimos utilizar los casos de mayor dificultad. 

En un primer momento queriamos evaluar cuantos ciclos y cuantas hormigas debiamos utilizar. 
Para esto dejamos fijo el número de ciclos y fuimos variando la cantidad de hormigas. 
Este primer resultado nos llamo la atención, ya que al incremetar el numero de hormigas, no se modificaba el tiempo 
promedio utilizado para encontrar la respuesta asi como tampoco la cantidad de 
ciclos generales usados para encontrar la solucion.



Una segunda prueba fue variar el factor de evaporación, pensando que a mayor nivel de evaporación, 
menor la incidencia de las hormigas en el algoritmo y por lo tanto mayor seria el tiempo, o inclusive que no 
encontraria la solución.
Lo que hicimos fue setear el factor de evaporación en $0.8$ pensando que se 
degradaria la performance de nuestro algoritmo. Pero nuevamente nos encontramos 
con que el algoritmo no se vio afectado.


Con estos resultados, la prueba que hicimos fue eliminar directamente el uso de 
la feronoma del algoritmo, por lo tanto las hormigas no compartirian 
informacion. 


Sorprendentemente, el algoritmo no se vio afectado por estos cambios. Siempre 
encontro la solucion y en tiempos promedios similares.


















